\documentclass[10pt,a4paper]{article}
\usepackage[utf8]{inputenc}

% preco nefungujes dpc
% \usepackage[T1]{fontenc}
\usepackage{charter}	% font set: Charter Serif
\usepackage{tgadventor} % font set: tgadventor Sans Serif
\renewcommand*\familydefault{\sfdefault}

\parindent=0pt
\pagenumbering{roman}

\usepackage{graphicx}
\graphicspath{ {./images/} } 

\usepackage{xcolor}
\definecolor{background}{rgb}{0.95,0.95,0.95}
\definecolor{codegray}{rgb}{0.5,0.5,0.5}

\usepackage{listings}
\lstdefinestyle{mystyle}{
    backgroundcolor=\color{background},
    keywordstyle=\color{blue},
    basicstyle=\footnotesize,
    numberstyle=\tiny\color{codegray},
    numbersep=5pt,
    numbers=left,
    showspaces=false,
    basicstyle=\ttfamily\small,
    breaklines=true
}
\lstset{style=mystyle}

\usepackage[a4paper]{geometry}
\geometry{
left=2cm,
top=2.5cm,
bottom=2.5cm, 
right=2cm}

\usepackage{fancyhdr}
\pagestyle{fancy}
\fancyhf{}
\lhead{Štatistická práca - Skúmanie času stráveného na Netflixe}
\rhead{Gabriela Závodská}
\cfoot{\thepage}

\usepackage{hyperref}
\hypersetup{
    colorlinks=true,
    linkcolor=blue,
    filecolor=magenta,      
    urlcolor=blue,
    pdftitle={Overleaf Example},
    pdfpagemode=FullScreen,
    }

\begin{document}

\begin{center}
\begin{LARGE}
    \textbf{Štatistická práca}

    Skúmanie času stráveného na Netflixe
    
\end{LARGE}

\begin{large}

    \emph{Pravděpodobnost a statistika 1, LS 2021/2022}
\end{large}
\end{center}

\section*{Získanie dát a ich spracovanie}

Kompletné dáta som získala z môjho Netflix účtu, ktorý v sebe zahŕňa 4 profily, k nahliadnutiu sú v súbore \href{https://github.com/GabiZavod/StatisticsProject/blob/05f6d4e53e625da1b6ee0fa13046abbf4deb6e70/data/ViewingActivity.csv}{\verb|data/ViewingActivity.csv}. Tento súbor obsahuje veľa pre túto prácu nepodstatných údajov, preto som ich prečistila pomocou skriptíčok v pythone, ktoré sú v zložke \href{https://github.com/GabiZavod/StatisticsProject/blob/05f6d4e53e625da1b6ee0fa13046abbf4deb6e70/data/data_modification_scripts}{\verb|data/data\_modification\_scripts}. Výsledné .csv súbory rozdelené podľa jednotlivých profilov (nazývaných len ako \verb|Student1|,..., \verb|Student4|) doplnené o dátumy, v ktoré nebol Netflix sledovaný (manuálne zarovnané na rovnaký začiatočný a konečný dátum) sú v zložke \href{https://github.com/GabiZavod/StatisticsProject/blob/05f6d4e53e625da1b6ee0fa13046abbf4deb6e70/data_final}{\verb|data\_final}. Prvý stĺpec obsahuje dátum, druhý stĺpec obsahuje počet hodín stráveých sledovaním Netflixu v ten deň. Pre ilustráciu:

\begin{lstlisting}
2022-06-01,1.1919
2022-05-31,0
2022-05-30,0
2022-05-29,2.4011
2022-05-28,0.9231
2022-05-27,2.1519
2022-05-26,1.4458
\end{lstlisting}

\section*{Strávia študenti rovnaký čas na Netflixe počas a mimo skúškového?}
Cieľom tejto práce je overiť hypotézu, či študenti trávia rovnako času na Netflixe počas skúškového, ako mimo neho. Dáta budem skúmať dvomi spôsobmi, najprv všetky štyri profily dohromady a následne každý profil zvlášť.

\vspace{2mm}
Pre účely tejto práce budem za skúškové obdobie pokladať:

\hspace{3mm} • letné skúškové: 05-20 až 06-30

\hspace{3mm} • zimné skúškové: 01-08 až 02-14

Považujem za potrebné tu podotknúť, že študenti, ktorým jednotlivé profily patria, študujú na rôznych vysokých školách, skúškové obdobia sa preto môžu jemne líšiť, tieto rozdiely budem považovať za nepodstatné.

\vspace{2mm}
Označme:

\hspace{3mm} • $S_1,...,S_n \sim N(\mu_s, \sigma_s^2)$ časy strávené sledovaním Netflixu počas skúškových období

\hspace{3mm} • $M_1,...,M_m \sim N(\mu_m, \sigma_m^2)$ časy strávené sledovaním Netflixu mimo skúškových období

\vspace{2mm}
Nulová hypotéza $H_0: \mu_s = \mu_m$ (počas a mimo skúškového strávia šudenti priemerne rovnaký čas

\hspace{5.2cm}denne na Netflixe)

Alternatívna hypotéza $H_1: \mu_s < \mu_m$ (počas skúškového strávia študenti priemerne menej času denne

\hspace{6.1cm}na Netflixe ako mimo skúškového)

% Alternatívna hypotéza $H_2: \mu_s > \mu_m$ (počas skúškového strávia študenti priemerne viac času denne

% \hspace{6.1cm}na Netflixe ako mimo skúškového)

Hladina významnosti: $\alpha = 0.05$

\begin{small}
Poznámka: Správne by alternatívna hypotéza mala byť len nerovnosť, no keďže chcem zistiť aj to, kedy je sledovanosť vyššia, používam dve alternatívne hypotézy a využívam pritom schopnosť knižnice \verb|scipy| analyzovať dáta aj týmto spôsobom.
\end{small}

\vspace{2mm}
Na overenie hypotézy budem používať \textbf{dvojvýberový Walshov t-test}, keďže nič nenaznačuje, že rozptyly náhodných veličín by mohli byť rovnaké.

\subsection*{Časť 1: Všetci študenti}
Po rozdelení časov jednotlivých študentov na časy z obdobia skúškového a mimo neho a následným spojením týchto zoznamov dostávam dva zoznamy: \verb|skuskove_all| a \verb|mimo_all|. (potrebné skriptíčka v \verb|stats/all_stat.py|)

Následne prevediem t-test: 
\begin{lstlisting}[language=Python]
>>> stats.ttest_ind(skuskove_all, mimo_all, equal_var=False, alternative="less")

Ttest_indResult(statistic=-3.1844329401039104, pvalue=0.0007389556448342999)
\end{lstlisting}

Získaná p-value je menšia ako haldina vierohodnosti $\alpha$, preto môžem $H_0$ zamietnuť. Platí teda alternatívna hypotéza $H_1$, teda že študenti strávia na Netflixe menej času počas skúškového, ako mimo neho.

\subsection*{Časť 2: Každý profil zvlášť}
Pozrime sa teraz na jednotlivých študentov zvlášť:

\textbf{Student1}:

\begin{lstlisting}
>>> stats.ttest_ind(skuskove, mimo, equal_var=False, alternative="less")

Ttest_indResult(statistic=-3.8022382781536277, pvalue=7.995788759218318e-05)
\end{lstlisting}

\textbf{Student2}:

\begin{lstlisting}
>>> stats.ttest_ind(skuskove, mimo, equal_var=False, alternative="less")

Ttest_indResult(statistic=-5.185564583985727, pvalue=1.6338836876844316e-07)
\end{lstlisting}

\textbf{Student3}:

\begin{lstlisting}
>>> stats.ttest_ind(skuskove, mimo, equal_var=False, alternative="less")

Ttest_indResult(statistic=-0.5842829109856048, pvalue=0.27967840340112504)
\end{lstlisting}

\textbf{Student4}:

\begin{lstlisting}
>>> stats.ttest_ind(skuskove, mimo, equal_var=False, alternative="less")

Ttest_indResult(statistic=1.8422393338979608, pvalue=0.9668485613877612)
\end{lstlisting}

\vspace{2mm}
Z týchto výsledkov môžme usúdiť:

\hspace{3mm} • Pri prvých dvoch študentoch je $H_0$ zamietnuteľná, keďže p-value $< \alpha$

\hspace{3mm} • Pri treťom študentovi $H_0$ zamietnuť nemôžeme

%%% TODO: kedy vlastne môžem zamietať a čo čo vyjadruje?? %%%

\hspace{3mm} • Pri štvrtom študentovi $H_0$ nemôžeme zamietnuť, no keďže hodnota štatistiky prekračuje kritickú 

\hspace{6mm} hodnotu t-rozdelenia s $n+m-2$ stupňami voľnosti (čo je pre počet stupňov volnosti rastúci nad 

\hspace{6mm} všetky medze $=1.6448$), budem sa týmto študentom zaoberať ešte trochu.

\vspace{2mm}
Výška p-hodnoty pri štvrtom študentovi naznačuje, že stredné hodnoty sledovania Netflixu mimo a počas skúškového sú s vysokou pravdepodobnosťou rovnaké. Výsledná hodnota štatistiky ale naznačuje, že $H_0$ zamietnuť môžme.

Skúsim preto sformulovať inú alternatívnu hypotézu $H_2: \mu_s > \mu_m$.

Po overení $H_0$ s $H_2$ ako alternatívou dostávame:

\begin{lstlisting}
>>> stats.ttest_ind(skuskove, mimo, equal_var=False, alternative="greater")

Ttest_indResult(statistic=1.8422393338979608, pvalue=0.03315143861223883)
\end{lstlisting}

Z toho plynie, že $H_0$ je zamietnuteľná v prospech $H_2$.

\pagebreak
%\section*{Je čas strávený na Netflixe dobre modelovaný normálnym rozdelením?}
%Na úvod sa pozrime na histogramy:

%Merania $S_1,...,S_n$ zachytáva nasledujúci histogram:

%\begin{lstlisting}[language=Python]
%plt.hist(skuskove_all, 30, (0,12.7), color="violet")

%plt.xlabel("Time spent on Netflix")
%plt.ylabel("Frequency")
%plt.title("Histogram casov pocas skuskoveho")

%plt.show()
%\end{lstlisting}

%\includegraphics[height=9 cm, width =15cm]{images/histogram_pocas.png}

%Merania $M_1,...,M_n$ zachytáva nasledujúci histogram:

%\begin{lstlisting}[language=Python]
%plt.hist(mimo_all, 30, (0,12.7), color="violet")
%plt.show()
%\end{lstlisting}

%\includegraphics[height=9 cm, width =15cm]{images/histogram_mimo.png}

%V oboch prípadoch už naoko vidíme, že rozdelenie neodpovedá normálnemu a skôr pripomína exponenciálne.

\end{document}